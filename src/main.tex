\documentclass[a4paper, 11pt]{article}
\usepackage[portuguese]{babel}
\usepackage{comment} % enables the use of multi-line comments (\ifx \fi)
\usepackage{fullpage} % changes the margin
\usepackage[a4paper, total={7in, 10in}]{geometry}
\usepackage{amsmath,mathtools}
\usepackage{amssymb,amsthm}  % assumes amsmath package installed
\usepackage{float}
\usepackage{graphicx}
\graphicspath{{imagens/}}
\usepackage{xcolor}
\usepackage{mdframed}
\usepackage[shortlabels]{enumitem}
\usepackage{indentfirst,setspace}
\usepackage{hyperref}
\usepackage{pgfplots,tikz, pgfplotstable}
\pgfplotsset{compat=1.18, width=10cm}
\usetikzlibrary{positioning, arrows.meta, automata}
\hypersetup{
    colorlinks=true,
    linkcolor=blue,
    filecolor=magenta,
    urlcolor=blue!70!red,
    pdftitle={Exame}, %%%%%%   WRITE ASSIGNMENT PDF NAME %%%%%%%%
    pdfborder=0 0 0
}
\usepackage[most,many,breakable]{tcolorbox}

\definecolor{mytheorembg}{HTML}{F2F2F9}
\definecolor{mytheoremfr}{HTML}{00007B}

\tcbuselibrary{theorems,skins,hooks}
\newtcbtheorem{exercicio}{Exercício}
{%
    enhanced,
    breakable,
    colback = mytheorembg,
    frame hidden,
    boxrule = 0sp,
    borderline west = {2pt}{0pt}{mytheoremfr},
    sharp corners,
    detach title,
    before upper = \tcbtitle\par\smallskip,
    coltitle = mytheoremfr,
    fonttitle = \bfseries\sffamily,
    description font = \mdseries,
    separator sign none,
    segmentation style={solid, mytheoremfr},
}
{p}

% To give references for any problem use like this
% suppose the problem number is p3 then 2 options either
% \hyperref[p:p3]{<text you want to use to hyperlink> \ref{p:p3}}
%                  or directly
%                   \ref{p:p3}

\input{structure/letterfonts}
\input{structure/macros}
\setlength{\parindent}{1.25cm}
\onehalfspacing
\setlength{\parskip}{6pt}

\usepackage[explicit,compact]{titlesec}
\titleformat{\chapter}[block]
{\bfseries\huge}{\filright\huge\thechapter.}{1ex}{\huge\filright #1}
\titlespacing*{\chapter}{0pt}{-30pt}{30pt}
\titlelabel{\thetitle.\quad}

\usepackage[style=apa]{biblatex}
\addbibresource{main.bib}

\begin{document}

    \textsf{\noindent \large\textbf{Nome do Autor} \hfill \textbf{Título do Documento}\\
    E-mail: E-mail do Autor %\href{https://sigarra.up.pt/feup/pt/fest_geral.cursos_list?pv_num_unico=201906772}{up201906772@up.pt}
    \hfill Curso: Sigla do Curso\\
    \normalsize UC: Nome da Cadeira \hfill Data: \today}

    % Problem 1
\begin{exercicio}{%problem statement
    Exame 2018/2019 - Parte sem consulta
}{p1 % problem reference text
}
    Explique o conceito de distribuição temperada e como se definem/caracterizam analiticamente.
\end{exercicio}

\solve{
%Solution
    Uma distribuição $g(t)$ é o processo de associar a uma função arbitrária $\phi(t)$, de uma certa classe C, um número \Ng que pode assumir qualquer quantidade dependente de \phit e são representados por:
    \begin{equation*}
        N_g\left[\phi(t)\right] \equiv \integral g(t) \cdot \phi(t) \cdot dt
    \end{equation*}

    A classe C contém as funções que possuem derivadas de todas as ordens, e que tendem para zero mais rapidamente que qualquer potência de $t$, quando $t$ tende para infinito.
    Representa-se analiticamente por:
    \begin{equation*}
        \phi(t) \in C \, \text{ se } \,
        \begin{cases}
            \exists \phi^{(n)}(t) & \forall t,n \\
            \lim_{t \rightarrow +\infty} \left[t^n \cdot \phi(t)\right] = 0
        \end{cases}
    \end{equation*}
}

% Problem 2
\begin{exercicio}{%problem statement
    Exame 2018/2019 - Parte sem consulta
}{p2% problem reference text
}
    Explique e defina analiticamente a operação de convolução e represente graficamente a seguinte convolução: $\delta(t+4)*\text{sign}(t)*\delta(t-3)$.
\end{exercicio}

\solve{
%Solution
    \[
        \delta(t+4)*\text{sign}(t)*\delta(t-3) = \text{sign}(t+1)
    \]
    $\delta(t) \rightarrow$ unidade de convolução, logo só temos de transladar o sinal!
    \begin{figure*}[!ht]
        \centering
        \begin{tikzpicture}
            \draw[->] (-4,0) -- (4,0) node[right] {$x$};
            \draw[->] (0,-2.5) -- (0,2.5) node[above] {$y$};
            \foreach \x in {-3,-2,-1,1,2,3}
            \draw (\x,-0.1) -- (\x,0.1) node[above] {$\x$};
            \foreach \y in {-1,1}
            \draw (-0.1,\y) -- (0.1,\y) node[right] {$\y$};
            \draw (0,0) -- (0,0) node[below left] {0};
            \draw[domain=-3:-1,smooth,variable=\x,blue] plot ({\x},{-1});
            \draw[domain=-1:3,smooth,variable=\x,blue] plot ({\x},{1});
            \filldraw[black] (-1,-1) circle (2pt) ;
            \filldraw[black] (-1,1) circle (2pt) ;
            \draw[blue] (-1,-1) -- (-1,1);
        \end{tikzpicture}\label{fig:figure}
    \end{figure*}

    A operação de convolução pode ser definida por:
    \[
        \phi(t) = \phi_1(t) * \phi_2(t) = \integral \phi_1(\tau) \cdot \phi_2(t-\tau) \cdot d\tau
    \]
    este integral, conhecido como integral de convolução, define uma função ou distribuição de $\tau$, e é simultaneamente funcional de $\phi_1(t)$ e $\phi_2(t)$ para todos os valores.

}

    
    %\nocite{*}
    %\printbibliography

\end{document}
